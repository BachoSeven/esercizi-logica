\documentclass[a4paper]{article}

\usepackage[T1]{fontenc}
\usepackage{textcomp}
\usepackage[italian]{babel}
\usepackage{hyperref}
\usepackage{amsmath, amssymb, amsthm}
% for \lightning
\usepackage{stmaryrd}
\usepackage{geometry}
\usepackage{tikz-cd}

\hypersetup{
	colorlinks = true, % links instead of boxes
	urlcolor   = blue, % external hyperlinks
	linkcolor  = blue, % internal links
	citecolor  = red   % citations
}

\newcommand{\R}{\mathbb{R}}
\newcommand{\C}{\mathbb{C}}
\newcommand{\Q}{\mathbb{Q}}
\newcommand{\N}{\mathbb{N}}
\newcommand{\A}{\mathbb{A}}
\newcommand{\Z}{\mathbb{Z}}
\newcommand{\limplies}{\rightarrow}
\newcommand{\liff}{\leftrightarrow}

\renewcommand{\labelitemii}{$\circ$}
\renewcommand{\Im}{\operatorname{Im}}

\newcommand\numberthis{\addtocounter{equation}{1}\tag{\theequation}}

\newtheorem{theorem}{Theorem}[section]
\newtheorem{lemma}{Lemma}[section]

\theoremstyle{definition}
\newtheorem{definition}{Definition}[section]

\theoremstyle{definition}
\newtheorem{example}{Example}[section]

\theoremstyle{remark}
\newtheorem*{remark}{Remark}

\theoremstyle{definition}
\newtheorem{exercise}{Esercizio}[section]
\newtheorem*{exercise*}{Esercizio}

\title{Esercizi del Corso di Logica Matematica 2022/2023}
\author{Francesco Minnocci}
\begin{document}
\maketitle
\section{Calcolo Proposizionale}
\begin{remark}
	Nel seguito, i valori di verità \textbf{falso} e \textbf{vero} verrano indicati per comodità rispettivamente con i numeri $0$ ed $1$.
\end{remark}
\setcounter{exercise}{6}
\begin{exercise}
	Mostrare che \(\rho(A)=\min\left\{ n\in\N \,|\: A\in \mathcal{F}_n \right\} \).
\end{exercise}
Visto che \(A\) è una formula sul linguaggio \(\mathcal{L}\), esiste un \( n \in \N \) tale che \(A\in \mathcal{F}_n\), quindi per il principio del minimo è ben definito
\[\min\left\{ n\in\N \,|\: A\in \mathcal{F}_n \right\}.\]
Procediamo ora per induzione su \(n\):
\begin{itemize}
	\item Se \(n=0\), \(A\) è una variabile e quindi \(\rho\left( A \right) =0\) e questo è anche il minimo.
	\item Se \(A\in \mathcal{F}_{n+1}\), essendo \[ \mathcal{F}_{n+1}=\mathcal{F}_n\cup \{\lnot P,\; P \land Q ,\; P\lor Q,\; P \limplies Q,\; P \liff Q \,|\: P,Q \in
		\mathcal{F}_n\} ,\]si distinguono due casi:
		se \(A\in \mathcal{F}_{n}\) si conclude per ipotesi induttiva, mentre se \(A\in \mathcal{F}_{n+1}\setminus \mathcal{F}_n\) si ha che \(n+1\) è il minimo livello in cui
		compare \(A\), e quindi se mostriamo che \(\rho(A)=n+1 \) abbiamo concluso. Si pongono perciò altri due casi:
		\begin{itemize}
			\item \(A=\lnot B\): \(\rho\left( A \right) = \rho\left( B \right) + 1\) con \(B\in \mathcal{F}_n\), e quindi per ipotesi induttiva \(\rho(B) \leq n\), ma non può essere \(\rho(B)<n\) perché altrimenti
				\(A\in \mathcal{F}_n\), e quindi \(\rho(B)=n\) da cui \(\rho(A)=n+1\).
			\item \(A=B\;\Diamond\;C\) con \(\Diamond\in\{\land,\lor,\limplies,\liff\}\): \(\rho(A)=\max\{\rho(B),\rho(C)\}+1\) con \(B,C\in \mathcal{F}_n\), ma allora
				per ipotesi induttiva \(\max\{\rho(B),\rho(C)\}\leq n\), ma come nel caso precedente la disuguaglianza non può essere stretta, in quanto altrimenti si
				avrebbe \(A\in \mathcal{F}_n\), per cui anche qui \(\rho(A)=n+1\).
		\end{itemize}
\end{itemize}
\hfill\(\square\)
\begin{exercise}
	Mostrare che se \(\rho(A)=n\) e \(k_A\) è il numero di connettivi presenti in A, allora \[ n\leq k\leq 2^n -1.\]
\end{exercise}
Procediamo per induzione su \(n\):
\begin{itemize}
	\item Se \(\rho(A)=0\) allora \(A\) è una variabile e quindi \(k_A=0\), quindi la disuguaglianza vale.
	\item Se \(\rho(A)=n+1\), distinguiamo i soliti casi:
		\begin{itemize}
			\item \(A=\lnot B\): allora \(\rho(B)=n\) che per ipotesi induttiva implica \(n\leq k_B\leq 2^n-1\), e visto che \(k_A=k_B+1\) si ha \[n+1\leq k_A \leq
				2^n\leq 2^n + \left( 2^n - 1 \right) = 2^{n+1} - 1\]
			\item \(A=B\;\Diamond\;C\) con \(\Diamond\in\{\land,\lor,\limplies,\liff\}\): essendo \(\max\{\rho(B),\rho(C)\}=n\) e \(k_A=k_B+k_C+1\), utilizzando
				l'ipotesi induttiva per \(B\) e \(C\) otteniamo
				\[ n+1=\max\{\rho(B),\rho(C)\}+1\leq\max\{k_B,k_C\}+1\leq k_A,\]
				ed anche
				\[ k_A \leq 2\cdot\max\{k_B,k_C\}+1\leq 2\cdot\max\{2^{\rho(B)}-1,2^{\rho(C)}-1\}+1=2^{n+1}-1,\]
				da cui segue la tesi.
		\end{itemize}
\end{itemize}
\begin{exercise*}\phantomsection\label{unicità}
	Mostrare che qualsiasi due interpretazioni di un linguaggio \(\mathcal{L}\) che estendono un stesso $\mathcal{L}$-assegnamento coincidono.
\end{exercise*}
Per mostrarlo, conviene ripercorrere la dimostrazione effettuata in classe per l'esistenza di un interpretazione delle formule che estende una specifica
funzione \(\alpha:\text{Form}\left( \mathcal{L} \right) \to \{0,1\}\), accorgendosi che nei vari casi c'è una sola scelta
possibile; infatti, mostrando che nel passo induttivo la definizione di \(I_\alpha\left( A \right) \) è obbligata, restringendoci a formule costruite con \(
\lnot, \land, \lor\) (visto che \(\limplies,\liff\) si possono costruire a partire da essi):
\begin{itemize}
	\item \(A=\lnot B\): per induzione è ben definito $I_{\alpha}\left( B \right).$ Supponiamo che \(I_{\alpha}\left( B \right)=0\), allora se per assurdo ponessimo \(I_{\alpha}\left( A \right)=0\) non
		varrebbe che \[ I_{\alpha}\left( B \right) = 0 \iff I_{\alpha}\left( \lnot B \right) = 1,\] contraddicendo la definizione di interpretazione booleana; il caso
		\(I_{\alpha}\left( B \right) = 1\) è del tutto analogo.
	\item \(A=B\land C\): per ipotesi induttiva sono ben definiti $I_{\alpha}\left( B \right), I_{\alpha}\left( C \right) $. Per definizione di interpretazione booleana, deve
		valere che $$ I\left( A \right) =1 \iff I_{\alpha}\left( B \right) =I_{\alpha}\left( C \right)=1  ,$$
		ma allora anche qui non possiamo effettuare alcuna scelta nella definizione di $I_{\alpha}\left( A \right) .$
	\item \(A=B\lor C\): si procede in maniera simile al caso precedente, essendo $I_{\alpha}$ univocamente determinato dai valori di $I_{\alpha}$ su $B$ e $C$.
\end{itemize}
\setcounter{exercise}{11}
\begin{exercise}
	Dimostrare che per ogni funzione $\operatorname{f}: \{0, 1\}^{\{1,\cdots,n\}} \to \{0, 1\}$ esiste una formula $A$ nelle sole variabili $X_1, \cdots ,
	X_n$ che ha come tabella di verità $\tau_A=f$.
\end{exercise}
\setcounter{exercise}{13}
\begin{exercise}[Modus Ponens]
	Se $\models A\limplies B$ e $\models A$, allora $\models B$.
\end{exercise}
Sia $I$ una qualsiasi interpretazione delle $\mathcal{L}$-formule. Allora $I(A)=1$ e $I(A\limplies B)=1$, quindi per definizione di interpretazione booleana deve valere $I(B)=1$.
\begin{exercise}
Mostrare che le seguenti formule sono tautologie per ogni scelta della formula $A$:
\begin{align*}
	A\lor\lnot A\\
	A\limplies A\\
	A\liff \lnot\lnot A
.\end{align*}
\end{exercise}
Il risultato si evince dalle tabella di verità di tali formule(omettendo $\lnot\lnot A$ che è banalmente $A$):
\begin{displaymath}
\begin{array}{c|cccc}
A & \lnot A & A\lor\lnot A & A\limplies A & A\liff \lnot\lnot A\\
\hline
1 & 0 & 1 & 1 & T\\
0 & 1 & 1 & 1 & T\\
\end{array}
\end{displaymath}
\begin{exercise}[Assiomi di Lukasiewicz-Frege-Hilbert-Mendelson]
	Le seguenti formule sono tautologie:
	\begin{align}
	&A \limplies \left( B\limplies A \right) \label{t1} \\
	&\left( A \limplies \left( B \limplies C \right)  \right) \limplies \left( \left( A\limplies B \right) \limplies \left( A \limplies C \right)  \right) \label{t2} \\
	&\left( \lnot A\limplies\lnot B \right) \limplies \left( \left( \lnot A \limplies B \right) \limplies A \right) \label{t3}
.\end{align}
\end{exercise}
Riduciamo tali formule utilizzando le adeguate proprietà (commutative, associative, distributive e leggi di De Morgan) dei connettivi, e l'equivalenza logica fra $A \limplies B$ e $\lnot A \lor B$: le formule \eqref{t1}, \eqref{t2} e \eqref{t3} diventano rispettivamente
\setcounter{equation}{0}
\begin{align*}
	&\lnot A \lor \left( \lnot B \lor A \right) \\\equiv& \left( A\lor\lnot A  \right) \lor \lnot B, \numberthis\\\\
	&\lnot A \lor \left( \lnot B \lor C \right) \limplies \lnot \left( \lnot A \lor B \right)  \lor \left( \lnot A \lor C \right)  \\\equiv& \lnot \left( \lnot A\lor\lnot B\lor C
\right) \lor \left( A\land \lnot B \right) \lor \left( \lnot A \lor C\right) \\\equiv& \left( A\land B\land \lnot C \right)\lor \left( A\lor \lnot A \right) \land \left( \lnot
B \lor \lnot A\right) \lor C \\\equiv& \left( A\lor \lnot A \right)\lor \left( A\land B\land \lnot C \right) \land \left(  \lnot A \lor\lnot
B \lor C\right)\\\equiv&\left( A\lor\lnot A \right) \lor \left( D\land \lnot D \right),\numberthis\\\\
		       &\lnot\left( A\lor\lnot B \right) \lor \left( \lnot\left( A\lor B \right) \lor A \right) \\\equiv& \left( \lnot A\land B \right) \lor\left( \left( \lnot A \land\lnot B \right) \lor A
\right) \\\equiv&
\left( \lnot A\land B \right) \lor \left( \left( \lnot A\lor A \right) \land\left( \lnot B\lor A \right)  \right) \\\equiv& \left( (\lnot A \land B)\lor(\lnot A\lor A) \right) \land
\left( \left( \lnot A\land B \right) \lor \left( \lnot B\lor A \right)  \right) \\\equiv
															  &\left( A\lor\lnot A \right) \lor E\lor\left( E\land\lnot
															  E \right)\numberthis,
\end{align*}
dove ci siamo fermati con le manipolazioni appena abbiamo ottenuto una tautologia, considerando alla luce dell'esercizio 1.15 le formule del tipo $A\lor\lnot A$ tautologie, ed abbiamo posto $D:=\left( A\land B\land\lnot C \right) $ ed $E:=\left( \lnot A\land B \right) $.
\begin{exercise}
	 Dimostrare che non esistono tautologie $A$ dove compaiono soltanto i connettivi $\lor$ e $\land$.
\end{exercise}
Sia per assurdo $A$ una tautologia nelle sole variabili $X_1,\cdots,X_n$ in cui compaiono solo $\land$ e $\lor$.
Allora, fissata un'interpretazione $I$ tale che $I( X_1 )
=\cdots=I(X_n )=0$, da una facile induzione su $\rho(A)$ e dall'ispezione diretta delle tabelle di verità dei connettivi $\land$ e $\lor$ (per entrambi è impossibile ottenere $1$
da formule entrambe con valore $0$), concludiamo che $I( A ) =0$, contro l'ipotesi che A fosse una
tautologia.
\setcounter{exercise}{18}
\begin{exercise}
	Sia $A$ una formula in $\mathcal{L}$ nelle sole $n$ variabili proposizionali $X_1,\cdots,X_n $. Allora, A è una tautologia se e solo se $A\left( B_1\slash X_1,\cdots, B_n\slash X_n \right) $ è una
	tautologia per ogni scelta delle formule $B_1,\cdots,B_n$.
\end{exercise}
Per induzione su $n$: se $n=1$, data un'interpretazione $I_1$ definiamo l'interpretazione $I_2$ che coincide con $I_1$ su $X_2,\cdots,X_n$, e tale che
$I_2(X_1)=I_1(B_1)$, che è
unica per un esercizio \hyperref[unicità]{precedente}. Allora, si ha
$I_2(A)=I_1(A(B_1\slash X_1))=1,$ visto che $A$ è una tautologia.
Il passo induttivo segue poi facilmente, visto che possiamo effettuare le sostituzioni una per volta.

Mostriamo il viceversa per contrapositivo: se $A\left( B_1\slash X_1,\cdots, B_n\slash X_n \right)$ non è
contraddittoria per ogni scelta delle formule $B_1,\cdots,B_n$, allora è una tautologia, e quindi scegliendo $B_i=X_i$ per ogni $i$ otteniamo che $A$ è una tautologia.
\setcounter{exercise}{28}
\begin{exercise}
	Mostrare che ogni formula è equivalente ad una in cui compaiono solo i seguenti connettivi:
	\begin{enumerate}
		\item $\left\{ \lnot,\land \right\}$
		\item $\left\{ \lnot,\lor \right\}$
	\end{enumerate}
	Mostrare inoltre che non vale tale proprietà per i seguenti connettivi:
	\begin{enumerate}
		\item $\left\{ \land,\lor \right\}$
		\item $\left\{ \lnot,\liff \right\}$
		\item $\left\{ \liff,\lor \right\}$
	\end{enumerate}
\end{exercise}
\noindent
Sia $A$ una formula,

Inoltre, per l'esercizio X $\left\{ \land,\lor \right\} $ non possono avere la proprietà menzionata visto che in particolare una tautologia non può contenere solo tali connettivi;
\setcounter{exercise}{30}
\begin{exercise}
	Per ogni formula $A$ esiste una formula $B\equiv A$ nella quale compare soltanto il connettivo "entrambe false" $\star$.
\end{exercise}
Utilizzando l'esercizio 1.29, ci basta mostrare ad esempio che una formula scritta solo con i connettivi $\left\{ \lnot,\lor \right\}$ è equivalente ad una formula che usa solamente il connettivo $\star$: infatti, osserviamo che per
definizione di tale connettivo si ha, per ogni scelta delle formule $A,B$ :
\begin{itemize}
	\item $\lnot A\equiv A\star A$
	\item $A\lor B\equiv\lnot\left( A\star B \right) \equiv \left( A\star B \right) \star\left( A\star B \right) $
\end{itemize}
\section{Teorie Proposizionali}
\end{document}
