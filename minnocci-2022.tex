\documentclass[a4paper]{article}

\usepackage[T1]{fontenc}
\usepackage{textcomp}
\usepackage[italian]{babel}
\usepackage{hyperref}
\usepackage{amsmath, amssymb, amsthm}
% for \lightning
\usepackage{stmaryrd}
\usepackage{geometry}
\usepackage{tikz-cd}

\hypersetup{
	colorlinks = true, % links instead of boxes
	urlcolor   = blue, % external hyperlinks
	linkcolor  = blue, % internal links
	citecolor  = red   % citations
}

\newcommand{\R}{\mathbb{R}}
\newcommand{\C}{\mathbb{C}}
\newcommand{\Q}{\mathbb{Q}}
\newcommand{\K}{\mathbb{K}}
\newcommand{\N}{\mathbb{N}}
\newcommand{\A}{\mathbb{A}}
\newcommand{\Z}{\mathbb{Z}}
\newcommand{\limplies}{\rightarrow}
\newcommand{\liff}{\leftrightarrow}

\renewcommand{\labelitemii}{$\circ$}
\renewcommand{\Im}{\operatorname{Im}}

\newcommand\numberthis{\addtocounter{equation}{1}\tag{\theequation}}

\newtheorem{theorem}{Theorem}[section]
\newtheorem{lemma}{Lemma}[section]

\theoremstyle{definition}
\newtheorem{definition}{Definition}[section]

\theoremstyle{definition}
\newtheorem{example}{Example}[section]

\theoremstyle{remark}
\newtheorem*{remark}{Remark}

\theoremstyle{definition}
\newtheorem{exercise}{Esercizio}[section]
\newtheorem*{exercise*}{Esercizio}

\title{Esercizi del Corso di Logica Matematica 2022/2023}
\author{Francesco Minnocci}
\begin{document}
\maketitle
\section{Calcolo Proposizionale}
\begin{remark}
	Nel seguito, i valori di verità \textbf{falso} e \textbf{vero} verrano indicati per comodità rispettivamente con i numeri $0$ ed $1$.
\end{remark}
\setcounter{exercise}{6}
\begin{exercise}
	Mostrare che \(\rho(A)=\min\left\{ n\in\N \,|\: A\in \mathcal{F}_n \right\} \).
\end{exercise}
Visto che \(A\) è una formula sul linguaggio \(\mathcal{L}\), esiste un \( n \in \N \) tale che \(A\in \mathcal{F}_n\), quindi per il principio del minimo è ben definito
\[\min\left\{ n\in\N \,|\: A\in \mathcal{F}_n \right\}.\]
Procediamo ora per induzione su \(n\):
\begin{itemize}
	\item Se \(n=0\), \(A\) è una variabile e quindi \(\rho\left( A \right) =0\) e questo è anche il minimo.
	\item Se \(A\in \mathcal{F}_{n+1}\), essendo \[ \mathcal{F}_{n+1}=\mathcal{F}_n\cup \{\lnot P,\; P \land Q ,\; P\lor Q,\; P \limplies Q,\; P \liff Q \,|\: P,Q \in
		\mathcal{F}_n\} ,\]si distinguono due casi:
		se \(A\in \mathcal{F}_{n}\) si conclude per ipotesi induttiva, mentre se \(A\in \mathcal{F}_{n+1}\setminus \mathcal{F}_n\) si ha che \(n+1\) è il minimo livello in cui
		compare \(A\), e quindi se mostriamo che \(\rho(A)=n+1 \) abbiamo concluso. Si pongono perciò altri due casi:
		\begin{itemize}
			\item \(A=\lnot B\): \(\rho\left( A \right) = \rho\left( B \right) + 1\) con \(B\in \mathcal{F}_n\), e quindi per ipotesi induttiva \(\rho(B) \leq n\), ma non può essere \(\rho(B)<n\) perché altrimenti
				\(A\in \mathcal{F}_n\), e quindi \(\rho(B)=n\) da cui \(\rho(A)=n+1\).
			\item \(A=B\;\Diamond\;C\) con \(\Diamond\in\{\land,\lor,\limplies,\liff\}\): \(\rho(A)=\max\{\rho(B),\rho(C)\}+1\) con \(B,C\in \mathcal{F}_n\), ma allora
				per ipotesi induttiva \(\max\{\rho(B),\rho(C)\}\leq n\), ma come nel caso precedente la disuguaglianza non può essere stretta, in quanto altrimenti si
				avrebbe \(A\in \mathcal{F}_n\), per cui anche qui \(\rho(A)=n+1\).
		\end{itemize}
\end{itemize}
\hfill\(\square\)
\begin{exercise}
	Mostrare che se \(\rho(A)=n\) e \(k_A\) è il numero di connettivi presenti in A, allora \[ n\leq k\leq 2^n -1.\]
\end{exercise}
Procediamo per induzione su \(n\):
\begin{itemize}
	\item Se \(\rho(A)=0\) allora \(A\) è una variabile e quindi \(k_A=0\), quindi la disuguaglianza vale.
	\item Se \(\rho(A)=n+1\), distinguiamo i soliti casi:
		\begin{itemize}
			\item \(A=\lnot B\): allora \(\rho(B)=n\) che per ipotesi induttiva implica \(n\leq k_B\leq 2^n-1\), e visto che \(k_A=k_B+1\) si ha \[n+1\leq k_A \leq
				2^n\leq 2^n + \left( 2^n - 1 \right) = 2^{n+1} - 1\]
			\item \(A=B\;\Diamond\;C\) con \(\Diamond\in\{\land,\lor,\limplies,\liff\}\): essendo \(\max\{\rho(B),\rho(C)\}=n\) e \(k_A=k_B+k_C+1\), utilizzando
				l'ipotesi induttiva per \(B\) e \(C\) otteniamo
				\[ n+1=\max\{\rho(B),\rho(C)\}+1\leq\max\{k_B,k_C\}+1\leq k_A,\]
				ed anche
				\[ k_A \leq 2\cdot\max\{k_B,k_C\}+1\leq 2\cdot\max\{2^{\rho(B)}-1,2^{\rho(C)}-1\}+1=2^{n+1}-1,\]
				da cui segue la tesi.
		\end{itemize}
\end{itemize}
\begin{exercise*}\phantomsection\label{unicità}
	Mostrare che qualsiasi due interpretazioni di un linguaggio \(\mathcal{L}\) che estendono un stesso $\mathcal{L}$-assegnamento coincidono.
\end{exercise*}
Per mostrarlo, conviene ripercorrere la dimostrazione effettuata in classe per l'esistenza di un interpretazione delle formule che estende una specifica
funzione \(\alpha:\text{Form}\left( \mathcal{L} \right) \to \{0,1\}\), accorgendosi che nei vari casi c'è una sola scelta
possibile; infatti, mostrando che nel passo induttivo la definizione di \(I_\alpha\left( A \right) \) è obbligata, restringendoci a formule costruite con \(
\lnot, \land, \lor\) (visto che \(\limplies,\liff\) si possono costruire a partire da essi):
\begin{itemize}
	\item \(A=\lnot B\): per induzione è ben definito $I_{\alpha}\left( B \right).$ Supponiamo che \(I_{\alpha}\left( B \right)=0\), allora se per assurdo ponessimo \(I_{\alpha}\left( A \right)=0\) non
		varrebbe che \[ I_{\alpha}\left( B \right) = 0 \iff I_{\alpha}\left( \lnot B \right) = 1,\] contraddicendo la definizione di interpretazione booleana; il caso
		\(I_{\alpha}\left( B \right) = 1\) è del tutto analogo.
	\item \(A=B\land C\): per ipotesi induttiva sono ben definiti $I_{\alpha}\left( B \right), I_{\alpha}\left( C \right) $. Per definizione di interpretazione booleana, deve
		valere che $$ I\left( A \right) =1 \iff I_{\alpha}\left( B \right) =I_{\alpha}\left( C \right)=1  ,$$
		ma allora anche qui non possiamo effettuare alcuna scelta nella definizione di $I_{\alpha}\left( A \right) .$
	\item \(A=B\lor C\): si procede in maniera simile al caso precedente, essendo $I_{\alpha}$ univocamente determinato dai valori di $I_{\alpha}$ su $B$ e $C$.
\end{itemize}
\setcounter{exercise}{13}
\begin{exercise}[Modus Ponens]
	Se $\models A\limplies B$ e $\models A$, allora $\models B$.
\end{exercise}
Sia $I$ una qualsiasi interpretazione delle $\mathcal{L}$-formule. Allora $I(A)=1$ e $I(A\limplies B)=1$, quindi per definizione di interpretazione booleana deve valere $I(B)=1$.
\begin{exercise}
Mostrare che le seguenti formule sono tautologie per ogni scelta della formula $A$:
\begin{align*}
	A\lor\lnot A\\
	A\limplies A\\
	A\liff \lnot\lnot A
.\end{align*}
\end{exercise}
Il risultato si evince dalle tabella di verità di tali formule(omettendo $\lnot\lnot A$ che è banalmente $A$):
\begin{displaymath}
\begin{array}{c|cccc}
A & \lnot A & A\lor\lnot A & A\limplies A & A\liff \lnot\lnot A\\
\hline
1 & 0 & 1 & 1 & T\\
0 & 1 & 1 & 1 & T\\
\end{array}
\end{displaymath}
\begin{exercise}[Assiomi di Lukasiewicz-Frege-Hilbert-Mendelson]
	Le seguenti formule sono tautologie:
	\begin{align}
	&A \limplies \left( B\limplies A \right) \label{t1} \\
	&\left( A \limplies \left( B \limplies C \right)  \right) \limplies \left( \left( A\limplies B \right) \limplies \left( A \limplies C \right)  \right) \label{t2} \\
	&\left( \lnot A\limplies\lnot B \right) \limplies \left( \left( \lnot A \limplies B \right) \limplies A \right) \label{t3}
.\end{align}
\end{exercise}
Riduciamo tali formule utilizzando le adeguate proprietà (commutative, associative, distributive e leggi di De Morgan) dei connettivi, e l'equivalenza logica fra $A \limplies B$ e $\lnot A \lor B$: le formule \eqref{t1}, \eqref{t2} e \eqref{t3} diventano rispettivamente
\setcounter{equation}{0}
\begin{align*}
	&\lnot A \lor \left( \lnot B \lor A \right) \\\equiv& \left( A\lor\lnot A  \right) \lor \lnot B, \numberthis\\\\
	&\lnot A \lor \left( \lnot B \lor C \right) \limplies \lnot \left( \lnot A \lor B \right)  \lor \left( \lnot A \lor C \right)  \\\equiv& \lnot \left( \lnot A\lor\lnot B\lor C
\right) \lor \left( A\land \lnot B \right) \lor \left( \lnot A \lor C\right) \\\equiv& \left( A\land B\land \lnot C \right)\lor \left( A\lor \lnot A \right) \land \left( \lnot
B \lor \lnot A\right) \lor C \\\equiv& \left( A\lor \lnot A \right)\lor \left( A\land B\land \lnot C \right) \land \left(  \lnot A \lor\lnot
B \lor C\right)\\\equiv&\left( A\lor\lnot A \right) \lor \left( D\land \lnot D \right),\numberthis\\\\
		       &\lnot\left( A\lor\lnot B \right) \lor \left( \lnot\left( A\lor B \right) \lor A \right) \\\equiv& \left( \lnot A\land B \right) \lor\left( \left( \lnot A \land\lnot B \right) \lor A
\right) \\\equiv&
\left( \lnot A\land B \right) \lor \left( \left( \lnot A\lor A \right) \land\left( \lnot B\lor A \right)  \right) \\\equiv& \left( (\lnot A \land B)\lor(\lnot A\lor A) \right) \land
\left( \left( \lnot A\land B \right) \lor \left( \lnot B\lor A \right)  \right) \\\equiv
															  &\left( A\lor\lnot A \right) \lor E\lor\left( E\land\lnot
															  E \right)\numberthis,
\end{align*}
dove ci siamo fermati con le manipolazioni appena abbiamo ottenuto una tautologia, considerando alla luce dell'esercizio 1.15 le formule del tipo $A\lor\lnot A$ tautologie, ed abbiamo posto $D:=\left( A\land B\land\lnot C \right) $ ed $E:=\left( \lnot A\land B \right) $.
\begin{exercise}\label{connettivi}
	 Dimostrare che non esistono tautologie $A$ dove compaiono soltanto i connettivi $\lor$ e $\land$.
\end{exercise}
Sia per assurdo $A$ una tautologia nelle sole variabili $X_1,\cdots,X_n$ in cui compaiono solo $\land$ e $\lor$.
Allora, fissata un'interpretazione $I$ tale che $I( X_1 )
=\cdots=I(X_n )=0$, da una facile induzione su $\rho(A)$ e dall'ispezione diretta delle tabelle di verità dei connettivi $\land$ e $\lor$ (per entrambi è impossibile ottenere $1$
da formule entrambe con valore $0$), concludiamo che $I( A ) =0$, contro l'ipotesi che A fosse una
tautologia.
\setcounter{exercise}{18}
\begin{exercise}
	Sia $A$ una formula in $\mathcal{L}$ nelle sole $n$ variabili proposizionali $X_1,\cdots,X_n $. Allora, A è una tautologia se e solo se $A\left( B_1\slash X_1,\cdots, B_n\slash X_n \right) $ è una
	tautologia per ogni scelta delle formule $B_1,\cdots,B_n$.
\end{exercise}
Per induzione su $n$: se $n=1$, data un'interpretazione $I_1$ definiamo l'interpretazione $I_2$ che coincide con $I_1$ su $X_2,\cdots,X_n$, e tale che
$I_2(X_1)=I_1(B_1)$, che è
unica per un esercizio \hyperref[unicità]{precedente}. Allora, si ha
$I_2(A)=I_1(A(B_1\slash X_1))=1,$ visto che $A$ è una tautologia.
Il passo induttivo segue poi facilmente, visto che possiamo effettuare le sostituzioni una per volta.

Mostriamo il viceversa tramite contronominale se $A\left( B_1\slash X_1,\cdots, B_n\slash X_n \right)$ non è
contraddittoria per ogni scelta delle formule $B_1,\cdots,B_n$, allora è una tautologia, e quindi scegliendo $B_i=X_i$ per ogni $i$ otteniamo che $A$ è una tautologia.
\setcounter{exercise}{28}
\begin{exercise}\label{equivalenze}
	Mostrare che ogni formula è equivalente ad una in cui compaiono solo i seguenti connettivi:
	\begin{enumerate}
		\item $\left\{ \lnot,\land \right\}$
		\item $\left\{ \lnot,\lor \right\}$
		\item $\left\{ \lnot,\limplies \right\}$
	\end{enumerate}
	Mostrare inoltre che non vale tale proprietà per i connettivi $\left\{ \land,\lor \right\}$.
\end{exercise}
\noindent
Riduciamo ogni formula con un solo connettivo ad una logicamente equivalente in cui compaiano solamente i connettivi citati, dopodiché la tesi seguirà per induzione strutturale sulla
costruzione delle formule (facciamo uso delle leggi di De Morgan e delle altre proprietà note dei connettivi):
\begin{enumerate}
	\item
		\begin{itemize}
			\item $A\limplies B\equiv\lnot\left( A\land\lnot B \right)$
			\item $A\liff B\equiv \left( A\land B \right) \land \left( \lnot A \land\lnot B \right) $
			\item $A\lor B\equiv\lnot\left( \lnot A \land \lnot B \right) $
		\end{itemize}
	\item
		\begin{itemize}
			\item $A\limplies B\equiv\lnot A\lor B$
			\item $A\liff B\equiv \left( \lnot A\lor \lnot B \right) \lor \left( A \lor B \right) $
			\item $A\land B\equiv\lnot\left( \lnot A \lor\lnot B \right) $
		\end{itemize}
	\item
		\begin{itemize}
			\item $A\lor B\equiv\lnot A\limplies B$
			\item $A\land B\equiv \lnot\left( A\limplies \lnot B \right) $
			\item $A\liff B\equiv \left( A\limplies B \right) \land \left( B \limplies A \right) \equiv \lnot\left( \left( X \limplies Y \right) \limplies \lnot\left(
				Y\limplies X \right)  \right) $
		\end{itemize}
\end{enumerate}
Inoltre, per l'esercizio \hyperref[connettivi]{1.17} $\left\{ \land,\lor \right\} $ non possono avere la proprietà menzionata, visto che in particolare una tautologia non può contenere solo tali connettivi.
\setcounter{exercise}{30}
\begin{exercise}
	Per ogni formula $A$ esiste una formula $B\equiv A$ nella quale compare soltanto il connettivo "entrambe false" $\star$.
\end{exercise}
Utilizzando l'esercizio \hyperref[equivalenze]{1.29}, ci basta mostrare ad esempio che una formula scritta solo con i connettivi $\left\{ \lnot,\lor \right\}$ è equivalente ad una formula che usa solamente il connettivo $\star$: infatti, osserviamo che per
definizione di tale connettivo si ha, per ogni scelta delle formule $A,B$ :
\begin{itemize}
	\item $\lnot A\equiv A\star A$,
	\item $A\lor B\equiv\lnot\left( A\star B \right) \equiv \left( A\star B \right) \star\left( A\star B \right) $.
\end{itemize}
\section{Teorie Proposizionali}
\setcounter{exercise}{9}
\begin{exercise}
Supponendo che $T\models A\limplies B$, mostriamo che l'insoddisfacibilità di $T\cup\left\{ B \right\} $ implica l'insoddisfacibilità di $T\cup\left\{ A \right\} $, ma che non
vale il viceversa:
\end{exercise}
Mostriamo la contronominale: se $I\models T \cup\left\{A\right\}$, allora da $I(A)=I(A\limplies B)=1$ otteniamo (per definizione di interpretazione) che $I(B)=1$, per cui $I\models T\cup \left\{ B \right\} $.

Per vedere che non vale l'implicazione inversa, prendiamo $T=\left\{ \lnot A, B \right\} $ ed osserviamo che $T\models A\limplies B$, e $T\cup\left\{ A \right\} $ è insoddisfacibile,
mentre $T\cup\left\{ B \right\} $ è chiaramente soddisfacibile.
\setcounter{exercise}{31}
\begin{exercise}
	Sia $f: \N\to\N$ tale che $$\forall n\in\N~f(n)\neq n,$$
	e sia $\Gamma$ il grafo $(\N,E)$ con $E=\left\{ (n,f(n))\mid n\in\N \right\} $.
	Allora, il numero cromatico di $\Gamma$ è 3.
\end{exercise}
\noindent
Partiamo col formulare il linguaggio dei grafi $k$-colorati su un insieme fissato di vertici $V$ $$\mathcal{L}_V^k:=\left\{ X_{ab}\mid a\neq b\in V \right\}\cup\left\{ Y_a^i\mid a\in V,
i=1,\dots,k \right\}$$
Allora, dato un grafo $(V,E)$ la seguente $\mathcal{L}_V$-teoria è soddisfacibile esattamente quando il grafo è $k$-colorabile:
$$ T^k_{V,E}:=\left\{ \bigvee_{i=1}^k Y^i_a\mid a\in V\right\} \cup\left\{ X_{ab}\mid (a,b)\in E \right\} \cup \left\{ Y_a^i\limplies\bigwedge_{j\neq i}{\lnot Y^i_a} \right\} \cup
\left\{ Y_a^i\limplies\lnot Y_b^i\mid a,b\in V,i=1\dots k \right\}  .$$
Per compattezza proposizionale, $T_{V,E}^k$ è soddisfacibile se e solo se è finitamente soddisfacibile, che a sua volta equivale al fatto che ogni sottografo finito di $(V,E)$ sia
$k$-colorabile (basta prendere il massimo sottoinsieme finito $(V_0,E_0)$ di coppie di variabili coinvolte nella soddisfacibilità di un sottoinsieme finito della teoria).

Ora, mostriamo che ogni sottografo finito di $\Gamma$ ha numero cromatico $3$, da cui seguirà la tesi: chiaramente è possibile avere sottografi che non sono 2-colorabili (basta
prendere una $f$ non iniettiva); quindi basta mostrare
che sono tutti 3-colorabili, e preso un sottografo con $m$ vertici ed al più $m$ archi lo 3-coloriamo per induzione su $n\leq m$. Siano $C_1, C_2$ e $C_3$ i colori a disposizione. Per $n=2$ scegliamo
due colori diversi a caso; per il
passo induttivo procediamo come segue: visto che ci sono al più $n$ vertici, dalla formula $$ \sum_{i=1}^n{\operatorname{deg}(i)}=2\cdot\text{\# archi}\leq n$$ segue che esiste un
vertice con $\leq 2$ vertici, e dato che il grafo ottenuto rimuovendo tale vertice ha una 3-colorazione per ipotesi induttiva, riaggiungendo il vertice che si collega ad al più
altri due vertici ci basta scegliere un colore diverso da tali due vertici per ottenere una 3-colorazione del grafo originale.
\setcounter{exercise}{35}
\begin{exercise}
	Osserviamo preliminarmente che $X:=\left\{ 0,1 \right\} ^Y$ con la topologia prodotto indotta da quella discreta su $\left\{ 0,1 \right\} $ è compatto se e solo se vale la FIP (Finite Intersection Property) sulle famiglie di chiusi, e che basta controllare tale proprietà
	sui complementari degli aperti di base della topologia prodotto, che nel caso dello spazio $X$ sono prodotti indicizzati in $Y$ in cui solamente un numero finito di
	componenti è diverso
	dall'intero spazio $\left\{ 0,1 \right\}$.

	Ora, per collegarci al contesto delle teorie proposizionali notiamo che, considerando $\mathcal{L}=Y$, si ha che un interpretazione delle $\mathcal{L} $-formule è
	determinata da una funzione $\operatorname{f}: Y \to \left\{ 0,1 \right\}$, ovvero da un elemento di $\left\{ 0,1 \right\} ^Y$.

	Questo ci consente di formulare una corrispondenza $$\left\{\mathcal{L}\text{-teorie }T\right\}~\underset{i}{\leftarrow}\underset{j}{\rightarrow}~\left\{\mathcal{F}_T=\left\{ C_A \right\} _{A\in T}\right\},$$
	con quindi i $C_A$ della forma $$\prod_{Y\setminus S_A}{\left\{ 0,1 \right\} }\times\prod_{i\in S_A}{C_i},$$
	per degli insiemi $S_A$ finiti al variare delle formule $A\in T$, e con $C_i\subsetneq\left\{ 0,1 \right\}~\forall i\in S_A$.\\
	Infatti, data una teoria $T$ possiamo costruire la famiglia $j(T)$ dei chiusi $C_A=\{ f\in\left\{ 0,1 \right\}^Y\mid f\models A \}
	$ al variare delle formule $A \in T$.\\
	D'altronde, data una famiglia di chiusi di base $\mathcal{F}$ come sopra, possiamo associargli la teoria $i(\mathcal{F})$ delle formule corrispondenti alla tabella di
	verità descritta dalla parte finita del chiuso prodotto $C_A$ ($A\in T$ insieme di indici di $\mathcal{F}$).

	Con questa costruzione, vediamo subito l'equivalenza cercata: se supponiamo la compattezza proposizionale allora una famiglia di chiusi $F_T$ con la FIP ci dà una teoria finitamente
	soddisfacibile,	e quindi per compattezza un modello per la teoria, ovvero un elemento nell'intersezione globale della famiglia per costruzione di $j$; viceversa, se supponiamo che lo spazio
	$\left\{ 0,1 \right\}^Y$ sia compatto e consideriamo una teoria finitamente soddisfacibile, allora la famiglia di chiusi associata soddisfa la FIP, e quindi l'intersezione
	dell'intera famiglia produce un interpretazione
	della teoria (abbiamo utilizzato profusamente la corrispondenza fra formule e tavole di verità come esplicitata nell'esercizio \textbf{1.12}).
\end{exercise}
\begin{exercise*}
	Sia $\mathcal{U}$ un filtro su di un insieme $I$. Allora, le seguenti affermazioni sono equivalenti:
	\begin{itemize}
		\item $\mathcal{U}$ è un ultrafiltro
		\item $\mathcal{U}$ è un filtro massimale per inclusione
		\item $\forall C=\sqcup_{i=1}^r{C_i}\in\mathcal{U}~\exists! i\mid C_i\in\mathcal{U}$
	\end{itemize}
\end{exercise*}
Mostriamo rispettivamente $(1)\iff(2)$ ed $(1)\iff(3)$ :\\
\noindent
Supponiamo che $\mathcal{U}$ sia un ultrafiltro non massimale; allora esiste un filtro $\mathcal{F}$ che lo contiene strettamente, cioè $\exists
A\in\mathcal{F}\setminus\mathcal{U}$ ; ma allora $A\not\in\mathcal{U}\implies A^c\in\mathcal{U}\subset\mathcal{F}\implies \emptyset=A\cap A^c\in\mathcal{F}$, che è una
contraddizione.
Viceversa, se $\mathcal{U}$ è massimale, vogliamo mostrare che $$\forall A\subset I,~A\not\in\mathcal{U}\implies A^c\in\mathcal{U}:$$ supponiamo infatti che entrambi $A$ ed $A^c$
non stiano in $\mathcal{U}$ ; allora necessariamente uno dei due ha intersezione non banale con tutti gli elementi di $\mathcal{U}$ (altrimenti, se esistessero $B,
C\in\mathcal{U}$ con $B\cap A=\emptyset \land C\cap A^c=\emptyset$ avremmo che $\emptyset=B\cap C\in\mathcal{U}$, assurdo). Senza perdita di generalità, supponiamo che $A$ abbia
questa proprietà, e prendiamo il filtro generato dall'insieme $\mathcal{U}\cup \{A\}$ (notiamo che per quanto osservato tale insieme possiede la FIP, e quindi possiamo
effettivamente usarlo per definire un filtro come visto in classe). Ma allora, otteniamo un assurdo visto che tale filtro contraddice la massimalità di $\mathcal{U}$.\\

\noindent
Osserviamo preliminarmente che $(3)$ è equivalente, per induzione, al caso $r=2$.
Vediamo ora che $(1)\iff(3)$: se $\mathcal{U}$ è un ultrafiltro, allora dato $C=A\sqcup B$ con $C\in\mathcal{U}$ se $A \in\mathcal{U}$ non può esserlo anche $B$, visto che sono
disgiunti; e se nessuno dei due sta in $\mathcal{U}$ allora i loro complementari ci stanno (per la proprietà di ultrafiltro), ma allora ci sta anche l'intersezione dei
complementari, ovvero $C^c\in\mathcal{U}$, che è assurdo. Viceversa, se assumiamo che valga $(3)$ e per assurdo $\mathcal{U}$ non è un ultrafiltro, allora $\exists A\subset
I\mid~A,A^c\not\in \mathcal{U}$. Ma allora, $I=A\sqcup(A\setminus I)\in\mathcal{U}\overset{(3)}{\implies}$ uno fra $A$ ed $A^c$ sta in $\mathcal{U}$, assurdo.
\begin{exercise*}
	Dimostrare che $$ \text{PA} \vdash \forall x \exists k~x=2k\lor x=2k+1.$$
\end{exercise*}
Usiamo l'induzione al primo ordine su $$\varphi(x):=\exists k~x=2k\lor x=2k+1:$$
\begin{itemize}
	\item $\varphi(0)\equiv \exists k~0=2k\lor 0=2k+1$: $$\forall\mathfrak{M}\models \text{PA}~0\in M\implies\exists k\in M~k+k=0$$ (basta prendere $k=0$ ed usare l'assioma $\forall x~
		x+0=x$ con l'assegnamento $\rho(x)=0$.)
		Ora, se supponiamo che $\forall x~\varphi(x)$, ovvero $$ \forall x\exists \overline{k}~ 2 x=\overline{k}\lor x=2 \overline{k}+1,$$ allora vogliamo mostrare che $$
		\exists k~s(x)=2k\lor s(x)=2k+1 .$$ Infatti, procediamo per casi: se $x=2 \overline{k}\implies s(x)=s(x+0)=x+s(0)=2 \overline{k}+1$, quindi possiamo scegliere
		$k=\overline{k}.$ Se invece $x=2 \overline{k}+1,$ preso $k=\overline{k}+1$ abbiamo che $s(x)=x+1=(2\overline{k}+1)+1=2(\overline{k}+1)=2k.$
\end{itemize}
\begin{exercise*}
	Dimostrare che $$ \text{PA} \vdash \forall x \forall y~~y>x\limplies y \geq s(x)  .$$
\end{exercise*}
Ricordiamo che $$x<y \equiv \exists z \neq 0\mid~ x+z=y.$$ Ora, notiamo che
$$ y\geq s(x) \liff \exists z~y+z=s(x) ,$$
visto che con l'assioma $\forall x~x+0=x$ possiamo incorporare l'eventuale uguaglianza semplicemente rimuovendo la condizione $z\neq 0$ dalla definizione di uguaglianza.
Dopodichè, sfruttando l'assioma $$\forall x \forall y~x+s(y)=s(x+y),$$ e la definizione ricordata sopra otteniamo che $$\exists z\neq 0\mid~ s(x)=s(y+z)=y+s(z),$$ che mostra la
tesi.
\begin{exercise*}
	Dimostrare che $$ \text{PA}\vdash \forall x~x\neq s(x) .$$
\end{exercise*}
Mostriamolo per induzione al prim'ordine:
\begin{itemize}
	\item se $x=0$, allora dall'assioma $$ \forall x~s(x)\neq 0 $$ segue che $0\neq s(0)$. Inoltre, il passo induttivo è $$ \forall x \left( x\neq s(x) \limplies
	s(x)\neq s(s(x)) \right) ,$$ la cui contronominale è vera per l'assioma $$\forall x \forall y (s(x)=s(y)\limplies x=y).$$
\end{itemize}
\begin{exercise*}
	Sia $\mathcal{U}$ un ultrafiltro banale su  $I$,cioè $\mathcal{U}=\left\{ A\subset I\mid i_0\in A \right\} $ per un qualche $i_0\in I$ fissato. Allora, $$ \frac{\prod_{i\in
	I}{\mathfrak{M}_i}}{\mathcal{U}}\cong \mathfrak{M}_{i_0} .$$
\end{exercise*}
Consideriamo la proiezione di $\mathfrak{M}:=\frac{\prod_{i\in
I}{\mathfrak{M}_i}}{\mathcal{U}}$ su $\mathfrak{M}_{i_0}$; questa è un immersione perché, essendo l'ultrafiltro principale, le relazioni in $\mathfrak{M}$ valgono tra un insieme di
termini se e solo se
valgono quando proiettate su $\mathfrak{M}_{i_0}$. Inoltre, è chiaramente un omomorfismo suriettivo, e quindi essendo un'immersione suriettiva è anche un isomorfismo.
\begin{exercise*}
	Trovare una $k-$categorica ma non completa.
\end{exercise*}
Consideriamo la teoria $T$ dei gruppi abeliani divisibili senza torsione (ovvero spazi vettoriali su $\Q$), senza imporre alcuna condizione sulla dimensione di tali spazi. Questa è
$k$-categorica per ogni $k>\aleph_0$ visto che ogni modello $V$ (spazio vettoriale su $\Q$) con $$ |V|=k $$ ha necessariamente $\operatorname{dim}(V)=k ,$ e quindi due tali modelli
sono necessariamente isomorfi. Inoltre, mostriamo che $T$ non è completa: basta prendere il modello finito $\mathcal{V}=\{0\}\subset \Q$ e considerare l'enunciato $$ \tau:=\exists a \exists b~a\neq b
,$$ che è falso per ogni assegnamento su $\mathcal{V}$, quindi in particolare $T\not\models\tau$, ma vale anche che $T\not\models\lnot\tau$, visto che in qualsiasi  $\Q$-spazio di
dimensione $n\geq 1$ ha infiniti elementi distinti.
\begin{exercise*}
	Mostrare che la classe dei campi algebricamente chiusi è elementare ma non è definibile.
\end{exercise*}
Nel seguito, utilizzeremo due teoremi visti in classe: il primo è che una classe $k$ è definibile se e solo se sia $k$ che $k^c$ sono elementari; il secondo è che una classe è
elementare se e solo se è chiusa per ultraprodotti ed equivalenze elementari.
Ricordiamo il seguente risultato algebrico: $$\forall n~\exists \K_n\text{ campo non algebricamente chiuso dove ogni polinomio di grado }\leq n\text{ ha una radice}.$$
Per dimostrare che $k_{\text{ACF}}$ è elementare, esibiamo una teoria i cui modelli sono tutti e soli gli elementi di $k_{\text{ACF}}$: $$ T_{\text{ACF}}=T_{\text{CAMPI}}\cup\left\{ \forall a_0 \forall a_1
\exists x~a_0+a_1 x=0 \right\} \cup \left\{ \forall a_0,a_1,a_2 \exists x~a_0+a_1 x+a_2 x\cdot x=0 \right\} \cup \dots .$$
Ora, procediamo a
dimostrare che $k_{\text{ACF}}$ non è definibile mostrando che la classe complementare non è chiusa per ultraprodotti (e che quindi non è elementare).
Dopodiché, definiamo $$ \K:=\frac{\prod_{n\in \N}{\K_n}}{\mathcal{U}} ,$$ con $\mathcal{U}$ ultrafiltro non principale su $\N$ (osserviamo che quindi
$\mathcal{U}$ contiene il filtro di Frechet). Basta allora dire che il campo $\K$ è algebricamente chiuso per concludere, ma grazie al
Teorema di Łos, un assioma in $T_{\text{ACF}}\setminus T_{\text{CAMPI}}$ che parla dell'esistenza di uno zero dei polinomi di grado $\leq n$ è soddisfatto da $\K$ se e solo se
l'insieme degli indici delle componenti in cui è valido sta in $\mathcal{U}$; ma questo insieme sicuramente contiene gli $i\geq n$, perché grazie al fatto algebrico di cui sopra in
$\K_i$ il polinomio dato dalle componenti
$i$-esime ha uno zero, e quindi l'insieme degli indici su cui vale l'assioma è cofinito $\implies$ sta in Frechet $\implies$ sta in $\mathcal{U}$.
\begin{exercise*}
	Dimostrare che la teoria degli ordini densi ha esattamente 4 teorie complete che la estendono.
\end{exercise*}
Osserviamo innanzitutto che la teoria $T$ degli ordini densi senza estremi è completa, visto che ogni modello è isomorfo all'ordine dei razionali per un teorema di elementi di teoria degli
insiemi. Inoltre, se consideriamo le formule $$\exists x~\forall y~y\leq x$$ ed $$\exists x~\forall y~y\geq x,$$ la teoria $T$ è ottenuta esattamente unendo la teoria degli ordini
densi con la negazione di tali formule, ed inoltre le altre tre teorie coerenti che si ottengono aggiungendo alla teoria degli ordini densi le altre possibili combinazioni delle suddette due
formule (negandole o meno) sono tutte complete, visto che si può dimostrare tramite back-and-forth che sono $\aleph_0$-categoriche ed inoltre non hanno modelli finiti (basta essere
accorti nel trattare i minimi e/o i massimi nei casi in cui ci sono).
\begin{exercise*}
	Dimostrare che in $\left( \N,~< \right) $ elementi diversi hanno tipi diversi.
\end{exercise*}
Definiamo per induzione forte una formula che identifica univocamente ogni naturale $n\in\N$ :
\begin{itemize}
	\item $\phi_0(x):=\forall y~x\leq y$
	\item $\phi_{n+1}(x):=\forall y~\exists y_0\ldots\exists y_n~\bigwedge_{i=0}^n{\phi_i(y_i)}\land\bigwedge_{i=0}^n{x\neq y_i}\land\bigwedge_{i=0}^n{y\neq y_i}\land x\leq y$
\end{itemize}
Ora, $\phi_n(x)$ è una formula con unica variabile libera $x$ soddisfatta da $n\in\N$ e da nessun altro $m\neq n$, quindi tutti gli elementi di $(\N,<)$ hanno necessariamente tipi distinti.
\begin{exercise*}
	Dimostrare che in $(\Q,<,+)$ esistono esattamente 3 tipi.
\end{exercise*}
Innanzitutto notiamo che le formule
\begin{itemize}
	\item $\varphi_1:=x+x=x$
	\item $\varphi_2:=x+x<x$
	\item $\varphi_3:=x<x+x$
\end{itemize}
mostrano che i tipi di $0,$ i numeri negativi ed i numeri positivi sono distinti.
Ora, dati $a,b$ diversi da $0$ tali che $$ \varphi_i(x)[\frac{a}{x}] \land \varphi_i(x)[\frac{b}{x}] $$ con $i=1$ oppure $i=2$, osserviamo che c'è un isomorfismo di ordini da $\Q$
in sè stesso dato da $\phi(x)=x\cdot \frac{b}{a}$. Questo preserva i tipi ed in particolare $\phi(x+x)=\phi(x)+\phi(x).$ Questo mostra che ci sono esattamente tre tipi.
\begin{exercise*}
	Mostrare che se $f(\overline{x},y)$ è primitiva ricorsiva, allora $\sum_{i\leq y}{f}$ e $\prod_{i\leq y}{f}$ lo sono.
\end{exercise*}
Dette $f_1$ ed $f_2$ rispettivamente la funzione somma e la funzione prodotto di cui sopra, possiamo definirle ricorsivamente come segue: \begin{align*}
&f_1(\overline{x},0)=f(\overline{x},0)\\
&f_1(\overline{x},y+1)=(f(\pi_1,s\circ\pi_2)+\pi_3)\left(  \overline{x},y,f(\overline{x},y) \right) \\
&f_2(\overline{x},0)=f(\overline{x},0)\\
&f_2(\overline{x},y+1)=(f(\pi_1,s\circ\pi_2)\cdot\pi_3)(\overline{x},y,f(\overline{x},y),\end{align*}
che quindi mostra che $f_{1,2}$ sono primitive ricorsive, dove $\pi_i$ sono le proiezioni di $\N^3$ sull'$i$-esima componente.
\begin{exercise*}
	Mostrare che la funzione successore è definibile, ovvero il suo grafico è un sottoinsieme definibile di $\N^2$.
\end{exercise*}
La formula che mostra la tesi è la seguente: $$ \phi(x,y)=\exists u~ (\forall z~z\cdot u=z)\land (x+u=y) ,$$
ovvero $u$ simboleggia l'unità ed infatti si ha che $(\N,+,\cdot)\models\phi(a,b)\iff b=s(a).$
\end{document}
